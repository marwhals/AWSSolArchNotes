% This is a template for doing homework assignments in LaTeX

\documentclass{article} % This command is used to set the type of document you are working on such as an article, book, or presenation

\usepackage{geometry} % This package allows the editing of the page layout
\usepackage{amsmath}  % This package allows the use of a large range of mathematical formula, commands, and symbols
\usepackage{graphicx}  % This package allows the importing of images

\newcommand{\question}[2][]{\begin{flushleft}
        \textbf{Question #1}: \textit{#2}

\end{flushleft}}
\newcommand{\sol}{\textbf{Solution}:} %Use if you want a boldface solution line
\newcommand{\maketitletwo}[2][]{\begin{center}
        \Large{\textbf{AWS Solution Architect notes}Course Title} % Name of course here
        \vspace{5pt}
        
        \normalsize{Matthew Frenkel \today}        % Change to due date if preferred
        \vspace{15pt}
        
\end{center}}

\begin{document}
\section{Introduction}
\section{Supporting Courses}
\section{AWS Fundamentals}
\section{Identity and Access Management (IAM)}
\section{Simple Storage Service (S3)}
\section{Elastic Computer Cloud (EC2)}
\section{Elastic Block Storage (EBS) and Elastic File System (EFS)}
\section{Databases}
\section{Virtual Private Cloud (VPC) Networking}
\section{Route 53}
\section{Elastic Load Balancing (ELB)}
\section{Monitoring}
\section{High Availability and Scaling}
\section{Decoupling Workflows}
\section{Big Data}
\section{Serverless Architecture}
\section{Security}
\section{Automation}
\section{Caching}
\section{Governance}
\section{Migration}
\section{Exam Preparation}
\section{Conclusion}

\end{document}